% !TEX root = ../Thesis.tex

A proposal has been made by O'Gormann \emph{et al} \cite{OGorman2014} for a scalable surface code quantum computer based around the spins of donor qubits in silicon acting as data qubits and their read-out via an optically addressable mobile probe qubit, acting as a measurement qubit. Key questions raised in this proposal are what qubits are appropriate for the data and measurement qubits.
Illumination by laser light is known to reduce the relaxation times of electrons bound to donors in silicon. So, a key feature of any optically addressable measurement qubit would be that its read-out does not cause significant decoherence in the data qubits in the silicon lattice. This report examines the effect of laser illumination of various wavelengths and powers on the relaxation and decoherence times of electrons bound to phosphorus donors in silicon.
A further question is the suitability of various impurities in silicon for data qubits with one vital feature being their ability to withstand electrical noise which can cause decoherence via the Stark shift.
The Stark shift is tested on the selenium$^{+}$ impurity, expected to be particularly robust to electrical noise due to its status as a deep donor.
One final possible feature of the proposal would be a quantum memory based on nuclear spins in silicon, enabling the use of the longer coherence times of nuclei for data storage.
Natural silicon has a proportion of $^{29}$Si nuclear spins that normally serve to reduce coherence times of donors in silicon. They could however be used as a quantum memory through their hyperfine coupling to the electrons bound to donors in silicon as suggested by Wolfowicz \emph{et al} \cite{Wolfowicz2016a}. The Stark shift on the hyperfine interaction between the donor electrons and the $^{29}$Si nuclear spins presents a potential control method - allowing modification of the hyperfine coupling. To verify the potential of this proposal, this report details attempts to measure the Stark shift of the interaction between phosphorus donors in natural silicon and the $^{29}$Si nuclei.