% !TEX root = ../Thesis.tex

\chapter{Literature Review}
\label{chap:litRev}

Having introduced the background theory to the work discussed in this report, I turn to a more detailed examination of the literature surrounding this work.
I begin with the various relaxation and decoherence processes for spins bound to donors in silicon.
First discussing the mechanisms by which these processes occur I shall then move on to the effect of illumination on these mechanisms.
Following this I shall move on the the stark shift of donor spins in silicon, including a brief introduction to the principle before an examination of previous work performed to characterise this effect in the various species of donor in silicon.

\section{Illumination induced Decoherence}

\subsection{Mechanisms of Relaxation and Decoherence}

The concept of $T_1, T_2$ and $T_2^*$ time was introduced in section \ref{sec:relProc} as the characteristic relaxation, decoherence and dephasing respectively. 
I shall now focus on the established mechanisms by which these occur, looking mainly at relaxation and decoherence as these represent a permanent loss of information.

\subsubsection{Relaxation}

The relaxation time, $T_1$, determines the rate at which the polarisation of an ensemble of spins returns to Boltzmann equilibrium after being disturbed.
The first method of relaxation that should be considered is spontaneous emission: the emission of a photon into free space as the spin relaxes from excited to ground state. 
For a magnetic dipole this rate is slow in normal circumstances - thousands of seconds or more \cite{schweiger2001principles,Baranov2017}. 
Given that above millikelvin temperatures the rate of transverse relaxation for donor spins in silicon is significantly greater than this it can be neglected as a dominant mechanism.
\\
Spin-lattice relaxation occurs instead via the interaction with of the spin system with vibrations in the silicon lattice - phonons. 
There are three different phonon-mediated processes that affect the relaxation rate of spin systems with transition frequency $\omega_s$, all of which are well described in a 1961 paper by Orbach \cite{VanVleck1940,Orbach1961}.
\\
\textbf{Direct Process}
\\
\noindent\rule{\columnwidth}{1pt}
\begin{adjustwidth}{1.5cm}{}
The first of these is a \textbf{direct process}, whereby a single phonon with frequency equal to that of the spin transition (as given by equation \ref{eq:spinHam}) is emitted into the lattice. 
This rate ($\propto 1/T_1$) is proportional to the phonon density at $\omega_s$ and varies with spin transition frequency and temperature as: $\omega_s^4T^{-1}$.
\par
\par
\end{adjustwidth}
\textbf{Raman Process}
\par\noindent\rule{\columnwidth}{1pt}
\begin{adjustwidth}{1.5cm}{}
Whilst this rate dominates at lower temperatures, $<10$k, at temperatures with $k_BT \gg \hslash\omega_s$, a two phonon Raman transition becomes more efficient.
As the maximum phonon density is at much higher frequencies than $\omega_s$, the spin first absorbs a phonon at $\omega_{\text{max}}$ before emitting a phonon with $\omega = \omega_{\text{max}} + \omega_s$. 
This corresponds to the spin transitioning to a virtual energy level before rapidly transitioning back to the spin ground state.
For non-integer spin systems the rate due to this effect scales with temperature as: $T^{7}$.
\end{adjustwidth}
\textbf{Orbach Process}
\\
\noindent\rule{\columnwidth}{1pt}
\begin{adjustwidth}{1.5cm}{}
The Raman process involves a virtual energy level so is an inherently off-resonant effect. 
It is possible for the spin to transition to an actual excited state during the two phonon process. 
This is known as an Orbach process and is more efficient than either process above. 
It scales with temperature as: $\exp\left({-\Delta E/K_BT}\right)-1$.
\end{adjustwidth}

\subsubsection{Decoherence}

At high temperatures the spin-lattice relaxation rate tends to be the dominant process for spins in silicon.
However, at temperatures below $\approx 10$k other factors become limiting. 
In particular the decoherence or $T_2$ time becomes important. 
This is the process by which the spins lose phase coherence irreversibly. 
There are several mechanisms that contribute to $T_2$, detailed here.
\\
\textbf{Spectral Diffusion}
\\
The first of these is spectral diffusion and its impact is largely dependent on the sample used.
In natural silicon samples there is a relatively high concentration of Spin-$\frac{1}{2}$ $^{29}$Si nuclear spins ($4.7\%4$). 
Due to the large extent of the donor electron wavefunction, it is highly probable that a given donor will experience a hyperfine interaction with one or more of these nuclei.
Although the large difference between the electron and nuclear gyromagnetic ratios prevents a state exchange or 'flip-flop' interaction, nearby pairs of $^{29}$Si nuclei do have this interaction.
The rate of exchange is slow, ($\approx 100$Hz) and will cause a change in the hyperfine interaction with any nearby donor electrons.
This will cause the acquisition of phase differences between electrons over time.
As these changes are time dependent they are \textit{not} refocused by a Hahn echo sequence \cite{Wolfowicz2015a}. 
In purified silicon samples this effect is reduced to the point that it is no longer the limiting factor in spin coherence.
\\
\textbf{Instantaneous Diffusion}
\\
In the first experiments on spin coherence times on purified silicon, the increase in coherence time was found to be on the order 2 fold \cite{Gordon1958}.
The reason behind this limited increase was the high donor concentrations, leading to interaction between donors.
Spins close enough to interact with one another will experience slightly different magnetic fields and the random nature of donor distribution leads to this effect being inhomogeneous.
Once again, this will not be reversed by a Hahn echo sequence as two donors interacting with one another will both be flipped, meaning that the phase acquisition is unchanged.
This effect is dependent on the concentration of donors in a sample according to:

\begin{equation}
\frac{1}{T_{2}^{\text{ID}}} = C(2\pi\gamma_e)^2\frac{\pi}{9\sqrt{3}}\mu_0\hslash,
\label{eq:instDiff}
\end{equation}

where C is the donor concentration.
\\
At concentrations useful for ESR techniques and at sufficiently low temperatures instantaneous diffusion will always be a limiting factor in spin coherence times.
